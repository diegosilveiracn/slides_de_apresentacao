\documentclass[aspectratio=169]{beamer}

\usepackage[brazil]{babel}
\usepackage[utf8]{inputenc}
\usepackage[T1]{fontenc}

\usetheme{Madrid}

\setbeamertemplate{navigation symbols}{}

\title[Quem é o professor?]{Quem é o professor?}

\author[Diego S. C. Nascimento]{Diego Silveira Costa Nascimento}

\institute[IFRN]{
	Instituto Federal de Educação, Ciência e Tecnologia do Rio Grande do Norte\\
	Campus Natal -- Cidade Alta\\
	diego.nascimento@ifrn.edu.br
}

\date[\today]{\today}

\begin{document}

\begin{frame}[plain]
	\includegraphics[scale=0.2]{imagens/IFRN}
	\titlepage
\end{frame}

\logo{\includegraphics[scale=0.1]{imagens/IFRN}}

\AtBeginSection[]{
	\begin{frame}
		\frametitle{Sumário}
		\tableofcontents[currentsection]
	\end{frame}
}

\begin{frame}
	\frametitle{Curiosidades}

	\begin{itemize}
		\item Professor do IFRN desde 2011;
		\item Campi do IFRN: 
			\begin{itemize}
				\item Ipangua\c cu; 
				\item Nova Cruz;
				\item Natal -- Zona Norte;
				\item \structure{Natal -- Cidade Alta}.
			\end{itemize}
		\item Pai de Diana e Adam;		
		\item Natural de Maceió - Alagoas;
		\item Torcedor do CSA; e
		\item Jogador de Mega Drive.
	\end{itemize}
\end{frame}

\begin{frame}
	\frametitle{Formações}

	\begin{itemize}
		\item Bacharel em Informática -- Análise de Sistemas -- Administração (CESMAC);
		\item Licenciado em Educa\c cão Profissional e Tecnológica (UAB -- IFRN);
		\item Especialista em Tecnologia da Informação (UFC);
		\item Mestre em Informática Aplicada (Unifor);
		\item Doutor em Ciências da Computação (UFRN);
		\item Pós-doutor em Ciências da Computação (UFRN); e
		\item Pós-doutor em Engenharia Informática (UC).
	\end{itemize}
\end{frame}

\begin{frame}
	\frametitle{Certifica\c cões}

	\begin{itemize}
		\item Sun Certified Java Programmer (Sun Microsystems); e
		\item Team Kanban Practitioner (Kanban University).
	\end{itemize}
\end{frame}

\begin{frame}
	\frametitle{Áreas de Interesse}
	
	\begin{itemize}
        		\item Linguagens de Programação;
		\item Inteligência Artificial; e
		\item Informática na educa\c cão.
	\end{itemize}
\end{frame}

\begin{frame}
	\frametitle{Combinados}
	
	\begin{itemize}
        		\item Chegada em atraso;
		\item Saída da sala; e
		\item Prazo das atividades.
	\end{itemize}
\end{frame}

\begin{frame}
	\frametitle{Mensagem}

	\begin{center}
	``É melhor você tentar algo, vê-lo não funcionar e aprender com isso, do que não fazer nada.''
	\end{center}
	
	\begin{flushright}
	Mark Zuckerberg
	\end{flushright}
\end{frame}

\end{document}